\section{BACKGROUND}
\label{sec:background}

\subsection{Notation}
The following notation is used throughout the paper.
\begin{itemize}
 \item The set of real numbers is denoted by $\mathbb{R}$. Let $u$ and $v$ be two $n$-dimensional column vectors of real numbers, i.e. $u,v \in \mathbb{R}^n$, 
 their inner product is denoted as $u^\top v$, with ``$\top$'' the transpose operator.
\item Given $u \in \mathbb{R}^3$, $u \times$ denotes the skew-symmetric matrix-valued operator associated with the cross product in 
  $\mathbb{R}^3$.
%  \item Given a time function $f(t) \in \mathbb{R}^n$, its first and second order time derivative are denoted by $\dot{f}(t)$ and $\ddot{f}(t)$,
%  respectively. Given a function $f(\cdot)$ of several variables, its gradient
% w.r.t. some of them, say $x$, is the row vector denoted as $ \partial_x f$.
 \item The Euclidean norm of either a vector or a matrix of real numbers is denoted by $|\cdot |$.
%  \item $b_i \in \mathbb{R}^n$ denotes the column vector of $n$ zeros but the $\imath$th coordinate, which is equal to one. 
\item $I_n \in \mathbb{R}^{n \times n}$ denotes the identity matrix of dimension~$n$; 
$0_n \in \mathbb{R}^n$ denotes the zero column vector of dimension~$n$; $0_{n \times m} \in \mathbb{R}^{n \times m}$ denotes the zero matrix of dimension~$n \times m$.
\item The vectors $e_1,e_2,e_3$ denote the canonical basis of $\mathbb{R}^3$.
\item Let $\mathcal{I} = \{O;e_1,e_2,e_3\}$ denote a 
fixed inertial frame with respect to (w.r.t.) which the sensor's absolute orientation is measured. 
Let $\mathcal{S} = \{O';i,j,k\}$ denote a frame attached to the sensor, where the matrix $T := (i,j,k)$ is a rotation matrix
whose column vectors are the vectors
of coordinates of $i,j,k$ expressed in the basis of  $\mathcal{I}$. 
\item The sensor's orientation w.r.t. $\mathcal{I}$ is characterized by the rotation matrix $T$. Given a vector of coordinates $\bar{x} \in \mathbb{R}^3$
expressed w.r.t. $\mathcal{I}$, we denote with $x$ the same vector expressed w.r.t. $\mathcal{S}$, i.e. $\bar{x} = Tx$.
\item Given $A \in \mathbb{R}^{n \times m}$ and $B \in \mathbb{R}^{p \times q}$, we denote with $\otimes$ the Kronecker product $A \otimes B \in \mathbb{R}^{np \times mq}$.
\item Given $X \in \mathbb{R}^{m \times p}$, $\text{vec}(X) \in \mathbb{R}^{nm}$ denotes the column vector obtained by stacking the columns of the matrix~$X$. 
In view of the definition of $\text{vec}(\cdot)$, it follows that \begin{equation}\label{eq:kroneckerVec} \text{vec}(AXB) = \left( B^{\top} \otimes A \right) \text{vec}(X).\end{equation}

% \item For the sake of conciseness, a vector 
% \[{\bf{x}} = \bar{x}_1 {\bf{i_0}} + \bar{x}_2 {\bf{j_0}} + \bar{x}_3 {\bf{k_0}} = x_1 {\bf{i}} + x_2 {\bf{j}} +x_3 {\bf{k}}\] 
% is written as 
% \[{\bf{x}} = ({\bf{i_0}},{\bf{j_0}},{\bf{k_0}})\bar{x} = ({\bf{i}},{\bf{j}},{\bf{k}})x\] 
% where the two vectors of coordinates $\bar{x}, x \in \mathbb{R}^3$ represent the vector $\bf{x}$ when expressed either w.r.t. the inertial
% frame $\mathcal{I}$ or w.r.t. the sensor frame $\mathcal{S}$, respectively. As a consequence, 
% \[ \bar{x} = R x. \]
\end{itemize}

\subsection{Problem statement and assumptions}
We assume that the model for predicting the force-torque (also called wrench)  
% applied to the sensor 
from the raw measurements is an affine model, i.e. 
% \begin{equation}
% \rawval = \shapemat \hspace{0.2em} \senswrench -  {o_r}
% \end{equation} 
% Consequently, the output of the sensor can be computed as $\nrofsg$ 
%(with $m \geq 6$) 
% raw strain gauges values is :
\begin{equation}
\label{wrenchInSensorCoordinates}
w =  C ( r - o),
% =  {C} \left( \rawval -  {o_r} \right)
\end{equation} 
where
% \begin{itemize}
% \item 
$ {w} \in \mathbb{R}^{6}$ is the wrench exerted on the sensor expressed in the sensor's frame,
% \item 
$r \in \mathbb{R}^{\nrofsg}$ is the raw output of the sensor,
% \item 
$ {C} \in \mathbb{R}^{6 \times \nrofsg}$ is the
%full row rank 
invertible
calibration matrix, and
% \item 
${o} \in \mathbb{R}^6$ is the sensor's offset.
The calibration matrix and the offset are assumed to be constant.
% \end{itemize}

We assume that the sensor is attached to a rigid body
% mechanical structure --~such as a robot manipulator, 
% see Figure~\ref{photoOfTheIcubLegOrArmWhereYouSeeTheMachanincalChainAttachedToIt}~-- and that
% this structure posses $n$ degrees of freedoms. The associated configuration of this structure can be
% then characterized by an $n$-dimensional column vector $q \in \mathbb{R}^n$. 
% The mechanical structure has a total 
of 
(constant) mass $m~\in~\mathbb{R}^+$ and with a center of mass whose position
w.r.t. the sensor frame $\mathcal{S}$ is characterized the vector $c \in \mathbb{R}^3$.

% \[{\bf{OC}}(q) = ({\bf{i_0}},{\bf{j_0}},{\bf{k_0}})\bar{c}(q) = ({\bf{i}},{\bf{j}},{\bf{k}})c(q).\]
% \[\bar{c} = Tc,\]
% with $\bar{c},c \in \mathbb{R}^3$ the vector of the center of mass expressed w.r.t. the inertial and sensor frame, respectively.
The gravity force applied to the 
% structure 
body
is  given by 
\begin{IEEEeqnarray}{RCL}
 \label{eq:g}
 m\bar{g} = mTg ,
\end{IEEEeqnarray}
with $\bar{g},g \in \mathbb{R}^3$ the gravity acceleration expressed w.r.t. the inertial and sensor frame, respectively. The gravity acceleration $\bar{g}$ is constant, so the vector $g$ does not have a constant direction, 
but has a constant norm.

Finally, we make the following main assumption.

\hp{}{
 The raw measurements $r$ are taken for static configurations of the 
 rigid body
%  mechanical structure 
 attached to the sensor, i.e. the angular velocity of the frame $\mathcal{S}$ is always zero.
 Also, the gravity acceleration $g$ is measured by an accelerometer installed on the rigid body.
%  $\tfrac{d q}{dt} \equiv 0$. 
  Furthermore, no external force-torque, but the gravity force, acts on the rigid body. Hence
}
\begin{IEEEeqnarray}{RCL}
\label{eq:staticWrench}
w &=& 
M(m,c) g, \IEEEyessubnumber  \\
M(m,c) &:=& m
\begin{pmatrix}
I_3 \\
c \times 
\end{pmatrix}. \label{matrixM}  \IEEEyessubnumber 
% = 
% \mathbf{M} \hspace{0.2em}{}^s\mathbf{g}
% =
% \mathbf{M} \hspace{0.2em}
% {}^s\rawval_w {}^w \mathbf{g}
\end{IEEEeqnarray} 

\remark{}{ We implicitly assume that the accelerometer frame is aligned with the force-torque sensor frame. 
This is a convenient, but non necessary, assumption. 
In fact, if the relative rotation between the sensor frame $\mathcal{S}$ and the accelerometer frame is unknown, 
it suffices to consider the accelerometer frame as the sensor frame $\mathcal{S}$.
}

Under the above assumptions, what follows proposes a new method for estimating the sensor's offset $o$ 
and for identifying the sensor's calibration matrix $C$ without the need of removing the sensor from the 
hosting system.
