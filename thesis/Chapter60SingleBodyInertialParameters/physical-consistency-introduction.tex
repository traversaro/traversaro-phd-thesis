\section{INTRODUCTION}
A large part of existing robotic systems are modeled as a system of multiple rigid bodies. 
The knowledge of the dynamical characteristics of these rigid bodies is a key assumption of model-based control and estimation techniques. The dynamics of a rigid body, i.e. how the acceleration of a rigid body is related to the forces applied on it, is completely described by the mass distribution of the body in the 3D space. 
The mass distribution itself is completely described by 10 \emph{inertial parameters} \cite{hollerbach2008model}. These parameters may be available if a good Computer-Aided Design (CAD) model of the robot is available, but often such models are either not available, or the mass distribution of the rigid bodies in the robot changes during operation, as in the case of an end effector that grabs and heavy object. 

% Inertial parameters identification is important, both for estimating the inertial parameters of an unknown payload or more generally to estimate inertial parameters of the links of a robot, that may be either unknown due to mismatch between design (CAD) models or because they are changing in time (payload estimation in a sense is a particular kind of inertial parameter identification: in this case the link that is changing inertial properties is the end effector \cite{hollerbach2008model}).

Inverse robot dynamics models can be written linearly with respect to the inertial parameters of the rigid bodies composing the robot. Classical identification techniques \cite{hollerbach2008model,ayusawa2014identifiability} consider the parameters of each body to be an element of the Euclidean space $\mathbb{R}^{10}$. Exploiting this fact, the inertial parameters identification problem has been classically posed as a \emph{Linear Least Square} optimization problem \cite{hollerbach2008model}. The resulting problem is convenient from a computational point of view, but it neglects the fact that not all vectors in $\mathbb{R}^{10}$ can be generated by a physical rigid body, i.e. it is possible that some inertial parameters are identified even if no physical rigid body could generate them. 
% We call inertial parameters that can correspond to real rigid body \emph{fully physically consistent} inertial parameters.

A necessary condition for the inertial parameters to be generated by a physical rigid body was first proposed in \cite{yoshida1994}: the \emph{physical consistency} condition.  
% to be the positiveness of the rigid body mass and the positive definiteness of the 3D inertia matrix at the center of mass of the body. 
This condition is important for control purposes because it ensures, if it is valid for all the links of a robot, the positive definiteness and the invertibility of the joint mass matrix \cite{yoshida2000}. This property is a key assumption in proving the stability of model-based control laws. 
The \emph{physical consistency} has been enforced in identification of inertial parameters  using several techniques: \cite{yoshida2000,mata2005,gautier2013identification,gautier2013positive,sousa2014physical,jovic2015identification}. However this condition is not \emph{sufficient}: it is possible that some inertial parameters that respect this condition do not correspond to any physical body: in particular this condition does not encode the \emph{triangle inequalities} of the 3D inertia matrix \cite[Chapter 3]{wittenburg2007dynamics}, as it will be explained in the remainder of the paper.

The main contribution of this paper is a new necessary and \emph{sufficient} condition for the inertial parameters to be generated by rigid body: the \emph{full physical consistency} condition. We show that this condition implies the already proposed \emph{physical consistency} condition and that the triangle inequalities are respected. Furthermore, we propose a nonlinear optimization formulation that takes into consideration this constraint by using state of the art optimization techniques on non-Euclidean manifolds \cite{brossette2015humanoid}. The proposed optimization technique is validated with a rigid body inertial identification experiment on the arm of the iCub humanoid robot.

For the sake of simplicity, in this paper, we only consider the problem of identifying the inertial parameters of a single rigid body. However, the \emph{full physical consistency} condition and the optimization on manifolds are general contributions, that could be applied to the case of the identification of inertial parameters in generic multibody structures. 

% The paper is organized as follows. Section II presents the
% notations  used  in  the  paper and the background on rigid body dynamics.
% Section III details the proposed \emph{full physical consistency} condition, the proposed nonlinear parametrization of the inertial parameters that ensures that this condition is always satisfied and the optimization technique on the manifold of the proposed parametrization.
% Section IV describe the experiments used for validation.
% Remarks  and perspectives conclude the paper.