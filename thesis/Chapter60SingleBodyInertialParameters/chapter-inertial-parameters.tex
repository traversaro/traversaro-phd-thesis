\chapter{Rigid Body Inertial Parameters Identification}
\label{ch:inertialParameters}
\graphicspath{{Chapter60SingleBodyInertialParameters/Figs/}}

\section{Introduction}
A large part of existing robotic systems are modeled as a system of multiple rigid bodies. 
The knowledge of the dynamical characteristics of these rigid bodies is a key assumption of model-based control and estimation techniques, such as the one presented in Chapter~\ref{chap:extForceAndJntTorqueEstimation}. The dynamics of a rigid body, i.e. how the acceleration of a rigid body is related to the forces applied on it, is completely described by the mass distribution of the body in the 3D space. 
The mass distribution itself is completely described by 10 \emph{inertial parameters} \citep{handbookident}. These parameters may be available if a good Computer-Aided Design (CAD) model of the robot is available, but often such models are either not available, or the mass distribution of the rigid bodies in the robot changes during operation, as in the case of an end effector that grabs and heavy object. 

% Inertial parameters identification is important, both for estimating the inertial parameters of an unknown payload or more generally to estimate inertial parameters of the links of a robot, that may be either unknown due to mismatch between design (CAD) models or because they are changing in time (payload estimation in a sense is a particular kind of inertial parameter identification: in this case the link that is changing inertial properties is the end effector \citep{hollerbach2008model}).

 Inverse robot dynamics models can be written linearly with respect to the inertial parameters of the rigid bodies composing the robot. Classical identification techniques \citep{handbookident,ayusawa2013} consider the parameters of each body to be an element of the Euclidean space $\mathbb{R}^{10}$. Exploiting this fact, the inertial parameters identification problem has been classically posed as a \emph{Linear Least Square} optimization problem \citep{handbookident}. The resulting problem is convenient from a computational point of view, but it neglects the fact that not all vectors in $\mathbb{R}^{10}$ can be generated by a physical rigid body, i.e. it is possible that some inertial parameters are identified even if no physical rigid body could generate them. 
% We call inertial parameters that can correspond to real rigid body \emph{fully physically consistent} inertial parameters.

A necessary condition for the inertial parameters to be generated by a physical rigid body was first proposed in \citep{yoshida1994}: the \emph{physical consistency} condition.  
% to be the positiveness of the rigid body mass and the positive definiteness of the 3D inertia matrix at the center of mass of the body. 
This condition is important for control purposes because it ensures, if it is valid for all the links of a robot, the positive definiteness and the invertibility of the joint mass matrix \citep{yoshida2000}. This property is a key assumption in proving the stability of model-based control laws. 
The \emph{physical consistency} has been enforced in identification of inertial parameters  using several techniques: \citep{yoshida2000,mata2005,gautier2013industrial,gautier2013positive,sousa2014,jovic2015}. However this condition is not \emph{sufficient}: it is possible that some inertial parameters that respect this condition do not correspond to any physical body: in particular this condition does not encode the \emph{triangle inequalities} of the 3D inertia matrix \cite[Chapter 3]{wittenburg2007dynamics}, as it will be explained in the remainder of the chapter.

The main contribution of this paper is a new necessary and \emph{sufficient} condition for the inertial parameters to be generated by rigid body: the \emph{full physical consistency} condition. We show that this condition implies the already proposed \emph{physical consistency} condition and that the triangle inequalities are respected. Furthermore, we propose a nonlinear optimization formulation that takes into consideration this constraint by using state of the art optimization techniques on non-Euclidean manifolds \citep{brossette2015humanoid}. The proposed optimization technique is validated with a rigid body inertial identification experiment on the arm of the iCub humanoid robot.

For the sake of simplicity, in this chapter, we only consider the problem of identifying the inertial parameters of a single rigid body. However, the \emph{full physical consistency} condition and the optimization on manifolds are general contributions, that could be applied to the case of the identification of inertial parameters in generic multibody structures. 

The chapter is organized as follows. Section~\ref{sec:background-on-inertial-parameters} presents the
notations  used  in  the  chapter and the background on rigid body dynamics.
Section~\ref{sec:full-physical-consistency} details the proposed \emph{full physical consistency} condition, the proposed nonlinear parametrization of the inertial parameters that ensures that this condition is always satisfied and the optimization technique on the manifold of the proposed parametrization.
Section~\ref{sec:single-body-experimental-results} describe the experiments used for validation.

The notation used in the chapter is summarized in the next table.

\[
  \left[
      \begin{tabular}{@{\quad}m{.05\textwidth}@{\quad}m{.83\textwidth}}
        {\Huge \faInfoCircle} &
          \raggedright%
           \textbf{Notation used in Chapter~\ref{ch:inertialParameters}} \par
          \begin{tabular}{@{}p{0.24\textwidth}p{0.55\textwidth}@{}}
              $A$ & Inertial frame. \\
              $B$ & Body frame. \\
              $\bbM := \ls_B \bbM_B$ & 6D inertia matrix of body $B$ expressed in the $B$ frame.  \\
              $m$ & Mass of the rigid body. \\
              $c := \ls^B c$ & Center of mass of the rigid body, expressed in the $B$ frame. \\
              $I_B$ & 3D inertia of the body $B$, expressed with the orientation of the body frame $B$ and w.r.t. the origin of $B$. \\
              $\alpha^g := \ls^B \alpha^g_{A,B} \in \R^3$ & Angular velocity of the body expressed in body frame. \\ 
              $\omega := \ls^B \omega_{A,B} \in \R^3$ & Angular velocity of the body expressed in body frame. \\
          \end{tabular}
      \end{tabular}
    \right]
\]


%%%%%%%%%%%%%%%%%%%%%%%%%%%%%%%%%%%%%%%%%%%%%%%%%%%%%%%%%%%%%%%%%%%%%%%%%%%%%%%

\section{Background on Rigid Body Inertial Parameters}
\label{sec:background-on-inertial-parameters}

\subsection{Rigid Body Dynamics} 
% The generic position of a body-fixed point $p$ of the body in inertial frame
% is given as:
% $$
% p = \ls^A R_B r + p_B 
% $$

% Deriving this equation we get the expression for the generic velocity of a point in the body, as a function of its body coordinates $r$:
% \begin{equation}
% \label{eq:pointVel}
% \dot{p} = \omega \times r + \dot{p}_B 
% \end{equation}.

% The kinetic energy of the body is given by
% \begin{equation}
% \label{eq:enKin}
% E_k = \iiint\limits_{\mathbb{R}^3} \frac{1}{2} \dot{p}^2(r) \rho(r) dr . 
% \end{equation}

% Assuming that the rigid body is subject to a uniform gravitational potential field with a constant gravitational acceleration vector $g \in \mathbb{R}^3$ expressed in the A frame, the potential energy is given as:
% \begin{equation}
% \label{eq:enPot}
% U = \iiint\limits_{\mathbb{R}^3} e_3^T \dot{p}(r) \rho(r) dr . 
% \end{equation}

The Newton-Euler equations using the \emph{proper sensor acceleration} \eqref{eq:properSensorAccelerationNewtonEuler} are given by:
\begin{equation}
\label{eq:newtonEuler}
\bbM \alpha^g 
+ 
\begin{bmatrix} 0_{3 \times 1} \\ \omega \end{bmatrix}
\bar{\times}^* \bbM 
\begin{bmatrix} 0_{3 \times 1} \\ \omega \end{bmatrix}
= \phi
\end{equation}

where $\alpha^g := \alpha^g_{A,B} \in \R^6$ is the \emph{sensor proper} acceleration, 
$\omega := \ls^B \omega_{A,B} \in \R^3$ is the angular velocity of the body expressed in body frame, $\phi := \ls_B\phi_{B}$ is the net force-torque acting on the robot expressed in $B$ frame and
$\bbM \in \mathbb{R}^{6 \times 6}$ is the 6D inertia matrix (also known as \emph{spatial inertia} in \cite{featherstone2008}) expressed in body frame $B$
\begin{equation}
    \bbM = \begin{bmatrix} m 1_{3} & -m c^\wedge \\ 
                        m c^\wedge           & {I}_B 
        \end{bmatrix}.
\end{equation}
Where:
\begin{itemize}
    \item $m \in \mathbb{R}$ is the mass of the rigid body,
    \item $c \in \mathbb{R}^3 := \ls^B c$ is the center of mass of the rigid body, expressed in the frame $B$,
    \item $I_B \in \mathbb{R}^{3\times3}$ is the 3D inertia matrix of the rigid body, expressed with the orientation of frame $B$ and with respect to the frame $B$ origin. 
\end{itemize} 

\subsection{Inertial parameters}
The 6D inertia matrix is parametrized by 10 parameters, usually called the \emph{inertial parameters} of the rigid body \citep{handbookident}, that are defined as $\pi \in \mathbb{R}^{10}$:
\begin{equation}
\label{eq:inertialParametersVector}
    \pi = 
    \begin{bmatrix}
    m \\
    m c \\
    \operatorname{vech}(I_B)
    \end{bmatrix} .
\end{equation}

The product of a generic vector $\begin{bmatrix} v \\ \omega \end{bmatrix} \in \R^6$ by the 6D inertia matrix $\bbM$ can be written as a product of a matrix in $\R^{6 \times 10}$ for the vector of inertial parameters $\pi$:
\begin{equation}
  \bbM \begin{bmatrix} v \\ \omega \end{bmatrix} 
  =
  D\left(\begin{bmatrix} v \\ \omega \end{bmatrix}\right)
  \pi
  =
\begin{bmatrix}
v            & \omega^\wedge & 0_{3 \times 6} \\
0_{3\times1} & -v^\wedge     & \omega \bullet 
\end{bmatrix} \pi
\end{equation}

where the matrix $\omega \bullet$ is defined such that $\omega \bullet \operatorname{vech}({I_B}) = {I_B} \omega$:
\begin{equation}
\omega \bullet = \begin{bmatrix} \omega_x & \omega_y & \omega_z & 0 & 0 & 0 \\ 
                                    0 & \omega_x & 0 & \omega_y & \omega_z & 0 \\ 
                                    0 & 0 & \omega_x & 0 & \omega_y & \omega_z \end{bmatrix}.
\end{equation}

\begin{proposition}[Newton-Euler equations linearity in the inertial parameters]
The Newton-Euler equations \eqref{eq:newtonEuler} can be written linearly \citep{garofalo2013closed,handbookident} in the inertial parameters~\eqref{eq:inertialParametersVector}:
\begin{equation}
\label{eq:inertialRegressor}
Y(\alpha^g, \omega) \pi 
=
\bbM \alpha^g + 
\begin{bmatrix} 0_{3 \times 1} \\ \omega \end{bmatrix}
\bar{\times}^* \bbM 
\begin{bmatrix} 0_{3 \times 1} \\ \omega \end{bmatrix}
= \phi, 
\end{equation}
with:
\begin{equation}
  Y(\alpha^g,\omega) =  D \left(\alpha^g \right) + \begin{bmatrix} 0_{3 \times 1} \\ \omega \end{bmatrix}
\bar{\times}^* D\left( \begin{bmatrix} 0_{3 \times 1} \\ \omega \end{bmatrix} \right) .
\end{equation}
\end{proposition}

\subsection{Relationship between the inertial parameters and the density function}
The mass distribution of a rigid body in space is described by its density function:
\begin{equation}
\rho(\cdot): \mathbb{R}^3 \mapsto \mathbb{R}_{\ge 0} .
\end{equation}
The domain of this function is the points of body expressed in the body-fixed frame $B$.
We consider the density equal to zero for the points outside the volume of the rigid body, so we can define the domain of $\rho(\cdot)$ to be all the points in the 3D space $\mathbb{R}^3$.

The inertial parameters are obviously a functional of the density $\rho(\cdot)$, in particular, we can define the functional $\pi_{d}(\cdot) : (\mathbb{R}^3 \mapsto \mathbb{R}_{\ge 0}) \mapsto \mathbb{R}^{10}$ that maps the density function to the corresponding inertial parameters: 
\begin{IEEEeqnarray}{rCl}
\label{eq:pid}
   \pi_{d}(\rho(\cdot)) &=&
    \begin{bmatrix}
    m(\rho(\cdot)) \\
    mc(\rho(\cdot)) \\
    \operatorname{vech}(I_B(\rho(\cdot))) 
    \end{bmatrix} = \nonumber \\ &=& 
   \begin{bmatrix}
     \iiint\limits_{\mathbb{R}^3} \rho(r) dr \\
     \iiint\limits_{\mathbb{R}^3} r \rho(r) dr \\
     \operatorname{vech}\left(
     \iiint\limits_{\mathbb{R}^3} (r^\wedge)^T r^\wedge \rho({r}) d{r} \right)
           \end{bmatrix} .
\end{IEEEeqnarray}

% \subsubsection{Mass of the body}
% The mass of the body is the integral of the body density over the space:
% \begin{equation}
% \label{eq:massDensity}
% \iiint\limits_{\mathbb{R}^3} \rho(r) dr = m 
% \end{equation}

% \subsubsection{First moment of mass}
% The \emph{first moment of mass} $mc$ is then defined as:
% \begin{equation}
% mc = {\iiint\limits_{\mathbb{R}^3} r \rho(x) dr}
% \end{equation}

% The center of mass is then defines simply as the division between the first moment of mass and the mass:
% \begin{equation}
%     c = \frac{mc}{m}
% \end{equation}

% \subsubsection{Inertia matrix with respect to the body frame origin}
% The 3D inertia matrix of the body is given by:
% $$
% I_B = \iiint\limits_{\mathbb{R}^3} S^T(r) r^\wedge \rho(r) dr = \iiint\limits_{\mathbb{R}^3} (r^T r 1_{3} - r r^T) \rho(r) dr
% $$

% It easy to very that the 3D inertia matrix is symmetric 
% $$
% I_B^T = \iiint\limits_{\mathbb{R}^3} (S^T(r) r^\wedge)^T \rho(r) dr = \iiint\limits_{\mathbb{R}^3} S^T(r) r^\wedge \rho(r) dr = I_B
% $$

% \subsubsection{Inertia matrix elements}
% \begin{align}
% I_B^{xx} = \iiint\limits_{\mathbb{R}^3} (y^2 + z^2) \rho(r) dr \\
% I_B^{xy} = I_B^{yx} = \iiint\limits_{\mathbb{R}^3} -xy~\rho(r) dr \\
% I_B^{xz} = I_B^{zx} = \iiint\limits_{\mathbb{R}^3} -xz~\rho(r) dr \\
% I_B^{yy} = \iiint\limits_{\mathbb{R}^3} (x^2 + z^2) \rho(r) dr \\
% I_B^{yz} = I_B^{yz} = \iiint\limits_{\mathbb{R}^3} -yz~\rho(r) dr \\
% I_B^{zz} = \iiint\limits_{\mathbb{R}^3} (x^2 + y^2) \rho(r) dr \\
% \end{align}

\subsection{3D Inertia at the Center of Mass and Principal Axes}
\label{subsec:principalAxis}

The 3D inertia at the center of mass is defined as

\begin{equation}
\label{eq:comInertiaDef}
I_C = \iiint\limits_{\mathbb{R}^3} ((r-c)^\wedge)^T (r-c)^\wedge \rho(r) dr .
\end{equation}

Exploiting the fact that $(\cdot)^\wedge$ is linear, the inertia matrix with respect to the center of mass can be written as: 
% \begin{align*}
% I_C = &\iiint\limits_{\mathbb{R}^3} \left( S^T(r) r^\wedge - S^T(r) S(c) - ( c^\wedge )^T r^\wedge + ( c^\wedge )^T S(c) \right) \rho(r) dr = \\
%  = &\iiint\limits_{\mathbb{R}^3}  S^T(r) r^\wedge \rho(r) dr -S^T \left( \iiint\limits_{\mathbb{R}^3}  r  \rho(r) dr \right)  S(c) + \\
% &-( c^\wedge )^T S \left( \iiint\limits_{\mathbb{R}^3}  r  \rho(r) dr \right) + ( c^\wedge )^T S(c) \iiint\limits_{\mathbb{R}^3}  \rho(r) dr = \\
% = &I_B -  m ( c^\wedge )^T S(c) -  m ( c^\wedge )^T S(c) + m ( c^\wedge )^T S(c) = \\ = &I_B -  m ( c^\wedge )^T S(c) =  I_B +  m S(c) S(c) 
% \end{align*}

\begin{equation}
I_C =  I_B +  m S(c) S(c) .
\end{equation}

This result is known as \emph{parallel axis theorem}. 

As $I_C$ is symmetric, it can be diagonalized with an orthogonal matrix $Q \in SO(3)$: 
\begin{equation}
I_C = Q \operatorname{diag}{(J)}  Q^T .
\end{equation}

Using \eqref{eq:comInertiaDef} the diagonal matrix $\operatorname{diag}{(J)}$ (with $J \in \mathbb{R}^3$) can be written as:
\begin{IEEEeqnarray}{rCl}
\operatorname{diag}{(J)} &=& \iiint\limits_{\mathbb{R}^3} Q^T ((r-c)^\wedge)^T (r-c)^\wedge Q \rho(r) dr = \nonumber \\
     &\iiint\limits_{\mathbb{R}^3}& ((Q^T (r-c))^\wedge)^T (Q^T(r-c))^\wedge \rho(r) dr .
\end{IEEEeqnarray}

The operation mapping the body point $r$ to $Q^T(r-c)$ can be interpreted as a change of 
reference frame, from the body frame $B$ to a frame $C$ (a \emph{principal axes} frame) whose origin is the center 
of mass of the body and whose orientation is one in which the $I_C$ matrix is diagonal.

By expressing with $\tilde{r} = Q^T \left( r-c \right)$ the generic point of the body expressed in the $C$ frame, and with $\tilde{\rho}(\tilde{r})$ the density with respect to the $C$ frame, we can write 
$\operatorname{diag}{J}$ as:
\begin{IEEEeqnarray}{rCl}
\operatorname{diag}{J} &=&   \iiint\limits_{\mathbb{R}^3}
(\tilde{r}^\wedge)^T \tilde{r}^\wedge \rho(\tilde{r}) d\tilde{r} .
\end{IEEEeqnarray}
The elements of the J vector are: 
\begin{IEEEeqnarray}{rCl}
J_{x} = \iiint\limits_{\mathbb{R}^3} ({\tilde{y}}^2 + {\tilde{z}}^2) \tilde{\rho}(\tilde{r}) dr' , \IEEEyessubnumber \\
J_{y} = \iiint\limits_{\mathbb{R}^3} (\tilde{x}^2 + \tilde{z}^2) \tilde{\rho}(\tilde{r}) d\tilde{r} , \IEEEyessubnumber \\
J_{z} = \iiint\limits_{\mathbb{R}^3} (\tilde{x}^2 + \tilde{y}^2) \tilde{\rho}(\tilde{r}) d\tilde{r} \IEEEyessubnumber .
\end{IEEEeqnarray}
We can write them as:
\begin{IEEEeqnarray}{rClrClrCl}
J_{x} &=& L_{y} + L_{z} , \quad
J_{y} &=& L_{x} + L_{z} , \quad
J_{z} &=& L_{x} + L_{y} .
\end{IEEEeqnarray}
\begin{IEEEeqnarray}{rClrCl}
L_{x} &=& \iiint\limits_{\mathbb{R}^3} {\tilde{x}}^2  \tilde{\rho}(\tilde{r}) d\tilde{r}, \quad
L_{y} &=& \iiint\limits_{\mathbb{R}^3} {\tilde{y}}^2  \tilde{\rho}(\tilde{r}) d\tilde{r}, 
\end{IEEEeqnarray}
\begin{IEEEeqnarray}{rCl}
L_{z} &=& \iiint\limits_{\mathbb{R}^3} {\tilde{z}}^2  \tilde{\rho}(\tilde{r}) d\tilde{r}.
\end{IEEEeqnarray}
Where $L_x , L_y , L_z$ are  the \emph{central second moments of mass} of the density $\tilde{\rho}(\tilde{r})$.
It is clear that the non-negativity of $\tilde{\rho}(\tilde{r})$ constraints $L = \begin{bmatrix} L_{x} & L_{y} & L_{z} \end{bmatrix}^T$ 
to be non-negative as well.
Furthermore, it is possible to see that the non-negativity of $L$ induces the \emph{triangular inequalities} on $\operatorname{diag}{\left(J\right)}$:
\begin{IEEEeqnarray}{lCrlCrlCr}
\label{eq:triangularInequalities}
 J_{x} &\leq& J_{y} + J_{z} , \quad 
 J_{y} \ &\leq& \ J_{x} + J_{z}, \quad 
 J_{z} \ &\leq& \ J_{x} + J_{y} .
\end{IEEEeqnarray}

% \begin{theorem}
% If the 3d inertia at the center of mass is positive definite, then all the 3d inertia at any arbitrary body fixed point are positive definite, i.e. 
% $$I_C \succeq 0 \implies I_B \succeq 0 ~ \forall B$$
% \end{theorem}

% \begin{proof}
% The proof is given in the Appendix.
% \end{proof}


% The mapping from the proposed parametrization to the classical inertial parameters vector ($\pi(p)$):
% \begin{align}
% m &= m \\
% mc &= m c \\
% \text{vech}(I_B) &= \text{vech}(R \operatorname{diag}{((P+P^T)L)} R^T - m S(c) S(c))
% \end{align}
% If we define as $p \in \mathbb{R}^{16}$ the following serialization of the inertial parameters:
% $$
% p 
% = 
% \begin{bmatrix}
% m \\
% c \\ 
% vec(R) \\
% L 
% \end{bmatrix}
% $$

%% We can then define the function $\pi(p)$ and its derivative is:
%% $$
%% \frac{\partial \pi}{\partial p}
%% =
%% \begin{bmatrix}
%% 1 & 0_{1 \times 3} & 0_{1 \times 9} & 0_{1 \times 3} \\
%% c & m I_3 & 0_{1 \times 9} & 0_{1 \times 3} \\
%% \text{vech}(-S(c)S(c))) & m \partial_c \text{vech}(-S(c)S(c))) &\partial_{\text{vec}{R}}  \text{vech}(I_B) & \partial_L  \text{vech}(I_B)\\ 
%% 
%% \end{bmatrix}
%% $$
%% 
%% where: 
%% \begin{align}
%%    \text{vech}(-S(c)S(c)) = 
%%    \text{vech}(c^T c I_3 - c c^T) =
%%    \begin{bmatrix} 
%%    c_y^2 + c_z^2 \\
%%    - c_x c_y \\
%%    - c_x c_z \\
%%    c_x^2 + c_z^2 \\
%%    - c_y c_z \\
%%    c_x^2 + c_y^2 
%%    \end{bmatrix}
%%    \\
%%     m \partial_c \text{vech}(-S(c)S(c))) =
%%     m 
%%     \begin{bmatrix} 
%%    0 & 2c_y & 2c_z \\
%%   - c_y & -c_x & 0 \\
%%    - c_z & 0    & -c_x \\
%%    2c_x &  0 &  2c_z \\
%%    0    & -c_z   & - c_y \\
%%   2c_x & 2c_y & 0 
%%    \end{bmatrix}
%% \end{align}
%% \begin{align}
%%    \partial_{R_{i,j}} \text{vech}(I_B) =
%%     \text{vech}(\Delta_{i,j} \operatorname{diag}{((P+P^T) L)} R^T + R \operatorname{diag}{((P+P^T)L)} \Delta_{i,j}^T) &
%% \end{align}
%% \begin{align}
%%     \partial_{L_i}  \text{vech}(I_B) = 
%%      \text{vech}(R \operatorname{diag}{((P+P^T) L )} R^T) &
%% \end{align}
%% Where $\Delta_{i,j}$ is the 3x3 matrix with all element zero except for the $i,j$ element and $e_i$ is the 3x1 vector with all element zero except for the $i$th element.

% Defining the $\frac{\partial \pi}{\partial p}$ is possible to convert all cost function regressor written for the classical linear parametrization in gradient of the new non-linear parametrization.

\subsection{Inertial Parameters Identification}
Assuming that $N$ values for $F$,$A_g$ and $V$ are measured, equation \eqref{eq:inertialRegressor} can be used to estimate $\pi$ solving the following optimization problem:
\begin{equation}
\label{eq:optimizationProblemLinear}
 \hat{\pi} =  \argmin_{\pi \in \mathbb{R}^{10}}\ ~ \sum_{i = 1}^N \left\| Y(\alpha^g_i, \omega_i)\pi - \rmf_i \right\|^2 . \\
\end{equation}

However, this optimization does not take into account the physical properties of the inertial parameters $\pi$. For this reason, the following definition was introduced.
\begin{definition}
\label{def:physicalConsistency}
A vector of inertial parameters $\pi$ is called \emph{physical consistent} \citep{yoshida1994,yoshida2000} if: 
\begin{IEEEeqnarray}{rClrCl}
m(\pi) &\geq& 0 , \qquad I_C(\pi) &\succeq& 0 .
\end{IEEEeqnarray}
\end{definition}
This condition has nice properties (it ensures that the matrix $M$ is always invertible), but is still possible to find some \emph{physical consistent} inertial parameters that can't be generated by a physical density. 

\section{Full Physical Consistency}
\label{sec:full-physical-consistency}
\subsection{Full physical consistency}
In this subsection, we propose a new condition for assessing if a vector of inertial parameters can be generated from a physical rigid body.  
We will show that all the constraints that emerge due to this \emph{full physical consistency} condition are due to the non-negativity on the density function.
\begin{definition}
\label{eq:fullDefinition}
A vector of inertial parameters $\pi^* \in \mathbb{R}^{10}$ is called \emph{fully physical consistent}  if: 
\begin{IEEEeqnarray}{rCl}
\exists\ \rho(\cdot) : \mathbb{R}^3 \mapsto \mathbb{R}_{\geq 0} ~ \text{s.t.} ~ \pi^* = \pi_d(\rho(\cdot)).
\end{IEEEeqnarray}
\end{definition}
This definition extends the concept of \emph{physical consistent} inertial parameters to include also all possible constraints of inertial parameters, such as the triangular inequalities \eqref{eq:triangularInequalities} of the diagonal elements of the inertia matrix.

\begin{lemma}
\label{lemma:lemma1}
If a vector of inertial parameters $\pi \in \mathbb{R}^{10}$ is \emph{fully physical consistent} if follows that it is \emph{physical consistent}, according to Definition \ref{def:physicalConsistency}. 
\end{lemma}
\begin{proof}
If $\pi$ is \emph{fully physical consistent}, then it follows that there exists $\rho(\cdot)$ such that the corresponding 3D inertia at the center of mass $I_C$ can be written as a function of $\rho(\cdot)$. The positive semi-definiteness of $m$ and $I_C$ then follows from the classical properties of mass and the inertia matrix of a rigid body, see for example subsection 3.3.3 of \cite{wittenburg2007dynamics}. 
\end{proof}

\begin{lemma}
If a vector of inertial parameters $\pi \in \mathbb{R}^{10}$ is \emph{fully physical consistent}, the associated inertia matrices at the body origin $I_B(\pi)$ and at the center of mass $I_C(\pi)$ respect the triangular inequalities \eqref{eq:triangularInequalities}. 
\end{lemma}

\begin{proof}
This lemma can be proved by writing $I_B$ or $I_C$ as a functional of the density function $\rho(\cdot)$, as in the proof of Lemma \ref{lemma:lemma1}. Once $I_B$ or $I_C$ are written as a functional of $\rho(\cdot)$, the demonstration that they respect the triangle inequality can be found in any rigid body mechanics textbook,  see for example subsection 3.3.4 of \cite{wittenburg2007dynamics}. 
\end{proof}
To get a hint of the demonstration of Lemma 2, consider that the diagonal elements of the 3D inertia matrix with respect to an arbitrary frame can still be written as the sum of two non-negative  \emph{second moments of mass}. The triangle inequality then arises in a way similar to the case of the inertia expressed in the principal axes.


\subsection{Full physical consistent parametrization of inertia parameters}
In this subsection, we introduce a novel nonlinear parametrization of inertial parameters that ensures the \emph{full physical consistency} condition. 

We choose to parametrize the mass as an element of the spaces of non-negative numbers $m \in \mathbb{R}_{\geq 0}$. 

The center of mass do not have any constraints on its location, so we choose to parametrize it as an element of the 3D space $c \in \mathbb{R}^3$. 

For parametrize the 3D inertia matrix ensuring the properties described in subsection \ref{subsec:principalAxis} we choose the \emph{second moments of mass} $L \in \mathbb{R}^3_{\geq 0}$ to be components of our parametrization. In the following, we will show how this choice ensures the \emph{full physical consistency} of the resulting inertial parameters. 

% The relation between $L = \begin{bmatrix} L^{xx} & L^{yy} & L^{zz} \end{bmatrix}^T \in \mathbb{R}^3$ and  $J = \begin{bmatrix} J^{xx} & J^{yy} & J^{zz} \end{bmatrix}^T \in \mathbb{R}^3$ can be written as:
% \begin{align}
% J = PL + P^T L = (P + P^T) L\\
% P =
% \begin{bmatrix}
% 0 & 1 & 0 \\
% 0 & 0 & 1 \\
% 1 & 0 & 0 
% \end{bmatrix}
% \end{align}
% where $P$ is a permutation matrix. 

% From the non-negativity of $\rho(r)$ we obtain as a necessary condition the non-negativity of $L_{x}$ , $L_{y}$, $L_{z}$. However, their non-negativity implies the non-negativity of $J_{x}$ , $J_{y}$, $J_{z}$, that implies the positive semi-definitess of $I_C$, i.e. the \emph{physical consistency} of the associated inertial parameters.
% \begin{IEEEeqnarray}{rCl}
% L_{x} , L_{y}, L_{z} \geq 0 \implies J_{x} , J_{y}, J_{z} \geq 0 \implies  I_C \succeq 0
% \end{IEEEeqnarray}

The inertial parameters of a rigid body can then be parametrized by an element $\theta \in \mathfrak{P} = \mathbb{R}_{\ge 0} \times \mathbb{R}^3 \times  SO(3) \times \mathbb{R}_{\ge 0}^3$. In particular the components of $\theta$ are:
\begin{itemize}
    \item $m \in \mathbb{R}_{\ge 0}$ the mass of the body 
    \item $c \in \mathbb{R}^3$ the center of mass of the body 
    \item $Q \in SO(3)$ the rotation matrix between the body frame and the frame of principal axis at the center of mass
    \item $L \in \mathbb{R}_{\ge 0}^3$ the second central moment of mass along the principal axes 
\end{itemize}

In other terms, there is a function $\pi_p(\theta) : \mathfrak{P} \mapsto \mathbb{R}^{10}$ that maps this new parametrization to the corresponding inertial parameters:
\begin{IEEEeqnarray}{rCCCl}
\label{eq:pip}
\pi_p(\theta) 
&=&
    \begin{bmatrix}
    m(\theta) \\
    mc(\theta) \\
    \operatorname{vech}\left(I_B(\theta)\right) \\
    \end{bmatrix}  
    &=& 
    \begin{bmatrix}
    m \\
    mc \\
    \operatorname{vech}\left( Q \operatorname{diag}{(PL)}  Q^T - m S(c) S(c) \right) \\
    \end{bmatrix} \nonumber
\end{IEEEeqnarray}
Where $P = \left[ \begin{matrix}
0 & 1 & 1 \\
1 & 0 & 1 \\
1 & 1 & 0 
\end{matrix} \right]$ is a matrix that maps $L$ to $J$.

\begin{theorem}
\label{thm:mainTheorem}
For each $\theta \in \mathfrak{P}$, there exists a density function $\rho(\cdot) : R^3 \mapsto R_{\geq 0}$ such that $\pi_d(\rho(\cdot)) = \pi_p(\theta)$, i.e. every  $\theta \in \mathfrak{P}$ generates \emph{fully physical consistent} inertial parameters.
\end{theorem}
\begin{proof}
\label{proof:mainTheorem}
% Given an element $\theta = (m,c,Q,L) \in \ipspace$, it always exists at least density function $\rho(\cdot): R^3 \mapsto R_{\geq 0}$ such that $\pi_p(\theta) = \pi_d(\rho(\cdot))$. 
We prove the statement in a constructive way: given an arbitrary element $\theta = (m,c,Q,L) \in \ipspace$ we build a density function $\rho(\cdot): \mathbb{R}^3 \mapsto \mathbb{R}_{\geq 0}$ such that $\pi_p(\theta) = \pi_d(\rho(\cdot))$. 
For example we can think of a cuboid of uniform unit density, with the center of the cuboid coincident with the center of mass of the inertial parameters (given by $c$), with the orientation of its symmetry axis aligned with the $C$ \emph{principal axes} frame defined by the $Q$ rotation matrix and the cuboid sides of lengths $2 d_x$ , $2 d_y$ and $2 d_z$, with:
\begin{equation}
\label{eq:cuboidSize}
    d = 
    \begin{bmatrix}
    d_x &
    d_y &
    d_z 
    \end{bmatrix}^{\top}
    = 
    \begin{bmatrix}
    \sqrt{3 \frac{L_x}{m}} &
    \sqrt{3 \frac{L_y}{m}} &
    \sqrt{3 \frac{L_z}{m}} 
    \end{bmatrix}^{\top}
\end{equation}


Its density function in the $C$ frame is given as:
\begin{equation*}
    \tilde{\rho}(\tilde{r}) = 
  \begin{cases} 
      \hfill 1 \hfill & \text{ if $-d \geq \tilde{r} \geq d$} \\
      \hfill 0 \hfill & \text{otherwise} \\
  \end{cases}
\end{equation*}
while the density function in the $B$ frame is given by:
\begin{equation}
  \label{eq:equivalentCuboidDensity}
    {\rho}({r}) = 
  \begin{cases} 
      \hfill 1 \hfill & \text{ if $-Qd+c \geq {r} \geq Qd+c$} \\
      \hfill 0 \hfill & \text{otherwise} \\
  \end{cases}.
\end{equation}

The density defined in \eqref{eq:equivalentCuboidDensity} and \eqref{eq:cuboidSize} can be seen as a function $\gamma(\cdot) : \ipspace \mapsto (\mathbb{R}^3 \mapsto \mathbb{R}_{\geq 0})$. The theorem is then demonstrated by using \eqref{eq:pid} and \eqref{eq:pip} to verify that:
$$
\pi_d(\gamma(\theta)) = \pi_p(\theta)
$$
is true $\forall~\theta \in \ipspace$.
\end{proof}

Using the parametrization presented in Theorem~\ref{proof:mainTheorem}, it is possible to recast the identification optimization problem \eqref{eq:optimizationProblemLinear} as:
\begin{IEEEeqnarray}{rCl}
\label{eq:optimizationProblemNonLinear}
 \hat{\pi} &=& \pi(\hat{\theta}) \\
% \hat{\theta} &=&  \argmin_{\theta \in \mathbb{R}_{\ge 0} \times \mathbb{R}^3 \times  SO(3) \times \mathbb{R}_{\ge 0}^3} ~ \sum_{i = 1}^N || Y(A^g, V) \pi(\theta) - F ||^2 
\hat{\theta} &=&  \argmin_{\theta \in \mathfrak{P}} \sum_{i = 1}^N \left\| Y(\alpha^g_i, \omega_i)\pi(\theta) - \rmf_i \right\|^2
\end{IEEEeqnarray}

The main advantage of \eqref{eq:optimizationProblemNonLinear} with respect to \eqref{eq:optimizationProblemLinear} is that thanks to Theorem \ref{thm:mainTheorem} the identified inertial parameters $\hat{\pi}$ are ensured to be \emph{fully physically consistent}. However, the optimization variable $\theta$ does not live anymore in a Euclidean space, because $\mathfrak{P}$ includes $SO(3)$, so to solve this optimization problem in this need to either modify or to exploit specific techniques related to the \emph{optimization on manifolds}, as we did in the experiments provided in Section~

\section{Experimental Results}
\label{sec:single-body-experimental-results}
\subsection{Optimization on Manifolds}
% In a nutshell, an n-dimensional manifold $\mathcal{M}$ is a space that locally looks like $\mathbb{R}^n$, \emph{i.e.}, for any point $x$ of $\mathcal{M}$ there exists a smooth map $\varphi_x$ between an open set of $\mathbb{R}^n$ and a neighborhood of $x$, with $\varphi_x(0) = x$.

For this chapter, we focus on $SO(3)$, a 3-dimensional manifold. As such, it can be parametrized \emph{locally} by $3$ variables, for example, a choice of Euler angles, but any such parametrization necessarily exhibits singularities when taken as a global map (e.g. gimbal lock for Euler angles), which can be detrimental to our optimization process.

For this reason, when addressing $SO(3)$ with classical optimization algorithms, it is often preferred to use one of the two following parametrizations:
\begin{itemize}
    \item unit quaternion, \emph{i.e.} an element $q$ of $\mathbb{R}^4$ with the additional constraint $\left\|q\right\| = 1$,
    \item rotation matrix, \emph{i.e.} an element $R$ of $\mathbb{R}^{3 \times 3}$ with the additional constraints $R^T R = I$ and $\det{R} \geq 0$. 
\end{itemize}

The alternative is to use optimization software working natively with manifolds~\citep{brossette2015humanoid}\citep{absil:book:2008} and solve
\begin{align}
\label{eq:finalProblem}
    \argmin_{\theta \in \mathbb{R}\times\mathbb{R}^3\times SO(3) \times \mathbb{R}^3} &\ \sum_{i = 1}^N \left\| Y(\rma^g_i, \rmv_i) \pi(\theta) - \rmf_i \right\|^2 \\
    \mbox{subj. to} &\ m \geq 0,\ L_x \geq 0,\ L_y \geq 0,\ L_z \geq 0
\end{align}

This alternative has an immediate advantage: we can write directly the problem \eqref{eq:optimizationProblemNonLinear} without the need to add any parametrization-related constraints. Because there are fewer variables and fewer constraints, it is also faster to solve. To check this, we compared the resolution of~\eqref{eq:optimizationProblemNonLinear} formulated with each of the three parametrizations (native $SO(3)$, unit quaternion, rotation matrix). We solved the three formulations with the solver presented in~\citep{brossette2015humanoid}, and the two last with an off-the-shelf solver (CFSQP~\citep{cfsqp:manual}), using the dataset presented later in this section. 
The formulation with native $SO(3)$ was consistently solved faster. We observed timings around $0.5$s for it, and over $1$s for non-manifold formulations with CFSQP. The mean time for an iteration was also the lowest with the native formulation (at least $30\%$ when compared to all other possibilities).

%Just like all $\mathbb{R}^{10}$ parametrization do not represent {\it fully physically consistent} inertial parameters, all quaternions or 3D matrices do not represent a valid element of $SO(3)$.
%Representing an element of $SO(3)$ which is a manifold of dimension 3, with an element $q$ of the dimension-4 quaternion space requires to enforce that the quaternion is unitary.
%Representing it with a 3D matrix $M$ requires to enforce that $M$ is symmetric, positive definite and of determinant one.
%Those parametrizations are widely used with classical optimization algorithm, but they require to implement the additional constraints stated above to ensure that the final result represents an element of $SO(3)$.

%Let's consider an optimization problem in which we search a symmetric positive definite matrix $M$ in the form $M = Q^TDQ$ with $Q\in SO(3)$ and $D$ is a real valued diagonal 3D matrix.
%The search space is $\mathcal{M} = \mathbb{R}^3\times SO(3)$.
%For $p\in\mathbb{R}^3$ we denote $D(p)$ the diagonal matrix with the elements of $p$ as diagonal values.
%For $p\in\mathbb{R}^3,\ p=[p_1,\ p_2,\ p_3]$ we denote $D(p) = \begin{pmatrix}
  %p_1 & 0 & 0 \\
  %0 & p_2 & 0 \\
  %0 & 0 & p_3
%\end{pmatrix}$.

%We want to minimize the difference between the sum of all eigenvalues of M denoted $\operatorname{eig}(M)$ and 42 while respecting a set of constraints $c$ with lower and upper bounds respectively $l$ and $u$.
%We denote $f(x) = (\operatorname{eig}(Q^TDQ) - 42)^2$.
%This problem writes as:

%\begin{align}
%\label{eq:pb}
%  \min_{x\in\mathcal{M}}\ & f(x) \\
%  \text{s.t.}\ & l \leq c(x) \leq u \nonumber
%\end{align}

%In that form, a classical optimization solver cannot solve this problem because it needs a parametrization of $\mathcal{M}$ over an Euclidean space.

%We parametrize the $SO(3)$ part of $x$ with a unit quaternion $q\in\{\mathbb{R}^4:\ \|q\|=1\}$, and denote $Q(q)$ the matrix associated with $q$, which is a rotation matrix iff $||q|| = 1$.
%With $x = \{q,p\}$.
%We can rewrite problem \eqref{eq:pb} so that it can be fed to any of-the-shelf optimization program as:

%\begin{align}
%\label{eq:pbQuat}
%  \min_{q\in\mathbb{R}^4,\ p\in\mathbb{R}^3} & (\operatorname{eig}(Q(q)^TD(p)Q(q)) - 42)^2 \\
%  \text{s.t.} & \|q\| = 1 \nonumber\\
%              & l \leq c(x) \leq u\nonumber 
%\end{align}

Working directly with manifolds has also an advantage that we do not leverage here, but could be useful for future work: at each iteration, the variables of the problem represent a fully physical consistent set of inertial parameters.
This is not the case with the other formulations we discussed, as the (additional) constraints are guaranteed to be satisfied only at the end of the optimization process. 
Having physically meaningful intermediate values can be useful to evaluate additional functions that presuppose it (additional constraints, external monitoring $\ldots$). 
It can also be leveraged for real-time applications where only a short time is allocated repeatedly to the inertial identification, so that when the optimization process is stopped after a few iterations, the output is physically valid.
With non-manifold formulations, at any given iteration, the parametrization-related constraints can be violated, thus, the variables might not lie in the manifold. It is then needed to project them on it. Denoting $\pi$ the projection (for example $\pi = \frac{q}{\left\|q\right\|}$ in the unit quaternion formulation), to evaluate a function $f$ on a manifold, we need to compute $f \circ \pi$. If further the gradient is needed, that projection must also be accounted for (\citep{bouyarmane2012humanoids} explains this issue in great details for free-floating robots).


%A potential issue with that approach is that the constraint $\|q\| = 1$ is guaranteed to be satisfied only at the solution. 
%At iteration $k$, from iterate $x_k$ an increment $p_k\in\mathbb{R}^7$ is computed and the next iterate is $x_{k+1} = x_k+p_k$.
%It is possible that $x_{k+1}$ violates some constraints, including the norm constraint, in which case, $q_{k+1}$ does not represent a rotation.
%In that case, there is no guarantee that $Q(q)^TD(p)Q(q)$ is diagonalisable in $\mathbb{R}$.
%Thus, the cost function is not correctly defined and the case of complex eigenvalues needs to be handled.
%Whereas if $\|q\| = 1$ the eigenvalues of $M(q,p)$ are $p_1$, $p_2$ and $p_3$.
%A usual method to deal with that issue is to normalize the quaternion in every function that uses it at each iteration, denoting $\pi(q) = \frac{q}{\|q\|}$, $f(x)$ and $c(x)$ become $f\circ\pi(x)$ and $c\circ\pi(x)$.
%And the gradients computations must take that projection into account, \citep{bouyarmane2012humanoids} explains that problem in great details for robotics problems with free-floating bases.
%If we use a $Q\in\mathbb{R}^{3\times3}$ parametrization of $SO(3)$, the additional constraints are $\{R^TR = I,\ \det(R)=1\}$ and the projection is the orthogonalization of $R$.

%So all in all this parametrization of the search manifold requires the optimization problem to have more variables (4 for quaternion or 9 for matrix), to add extra constraints to the problem and to modify the actual problem's functions to account for the projection on the search manifold.

%The inertial parameter identification suffers from similar problems with the inertia matrix needing to be symmetric and positive definite and in the formulation presented above, having non-negativity constraints on its eigenvalues to ensure to have a {\it fully physical consistent} inertia matrix.

%We propose to use optimization on manifolds to improve the formulation of our problem and avoid the aforementioned issues.

In this study, we use the same solver and approach as presented in \citep{brossette2015humanoid} which was inspired from \citep{absil:book:2008}.
%Considering the search space being a n-dimensional manifold $\mathcal{M}$. For any $x\in\mathcal{M}$ there exists of a smooth map $\varphi_x$ between the tangent space of $\mathcal{M}$ at $x$, $T_x\mathcal{M}$, and a neighborhood of $x$, with $\varphi_x(0) = x$. 
%$T_x\mathcal{M}$ can be identified to $\mathbb{R}^n$.
The driving idea of the optimization on manifold is to change the parametrization at each iteration. The problem at iteration k becomes:
\begin{IEEEeqnarray}{rClrCl}
  \min_{z_k \in \mathbb{R}^n}\ & f\circ\varphi_{x_k}(z) \quad
  \text{s.t.} \quad & c\circ\varphi_{x_k}(z) = 0 .
\end{IEEEeqnarray}
Then $x_{k+1} = \varphi_{x_k}(z_k)$ is guaranteed to belong to $\mathcal{M}$. The next iteration uses the same formulation around $x_{k+1}$.

%With that approach, we do not have any parametrization issues, do not need any additional constraints, do not need to project our iterates and have the minimum number of optimization parameters.

The smooth maps $\varphi_x$ are built-in and are used automatically by the solver while the user only has to implement the functions of the optimization problem without the burden of worrying about the parametrization.

\subsection{Experiments}

The iCub is a full-body humanoid with 53 degrees of freedom, thoroughly described in Section~\ref{sec:icub}.  For validating the presented approach, we used the six-axis force/torque (F/T) sensor embedded in iCub's right arm to collect experimental F/T measurements. We locked the elbow, wrist and hands joints of the arm, simulating the presence of a rigid body directly attached to the F/T sensor, a scenario similar to the one in which an unknown payload needs to be identified \citep{kubus2008line}. 
\begin{figure}[htb]
\begin{overpic}[width=1.0\textwidth,viewport=0 0 1235 742]{arm3.png}
\put(5,10){FT sensor}
\put(13,13){\vector(1,1){18}}
\put(38,50){Upper arm}
\put(43,49){\vector(0,-1){12}}
\put(65,45){Forearm}
\put(70,44){\vector(-1,-2){7}}
\end{overpic}
\caption{CAD drawing of the iCub arm used in the experiments. The used six-axis F/T sensor is visible in the middle of the upper arm link.}
\label{fig:cadArmSingleBody}
\end{figure}

We generated five 60 seconds joint positions paths in which the three shoulder joints were reaching random joint position using point to point minimum-jerk like trajectories. The point to point trajectory completion times were $10$s, $5$s, $2$s, $1$s and $0.5$s for the different paths. 
We played these joint paths on the robot and we sampled at $100$Hz the F/T sensors and joint encoders output. We filtered the joint positions and obtained joint velocities and accelerations using a Savitzky-Golay filtering of order 2 and with a windows size of $499$, $41$, $21$, $9$, $7$ samples. We used joint positions, velocities and accelerations with the kinematic model of the robot to compute $\rma^g$ and $\rmv$ of the F/T sensor for each time sample.
We removed the unknown offset from the F/T measurements using the offset removal technique described in \citep{traversaro2015situ}.
We then solved the inertial identification problem using the classical linear algorithm \eqref{eq:optimizationProblemLinear} and the one using the proposed \emph{fully physical consistent} parametrization \eqref{eq:finalProblem}.
We report the identified inertial parameters in Table \ref{table:results}. 
It is interesting to highlight that for slow datasets (trajectory time of $10$s or $5$s) the unconstrained optimization problem \eqref{eq:optimizationProblemLinear} results in inertial parameters that are not fully physical consistency. 
In particular, this is due to the low values of angular velocities and acceleration, that do not properly excites the inertial parameters, which are then \emph{numerically not identifiable}. 
The proposed optimization problem clearly cannot identify these parameters anyway, as the identified parameters are an order of magnitude larger than the ones estimated for faster datasets, nevertheless, it always estimates inertial parameters that are fully physical consistent. 
For faster datasets (trajectory time of $1$s or $0.5$s) the results of the two optimization problems are the same because the high values of angular velocities and accelerations permit to identify all the parameters perfectly. 
While this is possible to identify all the inertial parameters of a single rigid body, this is not the case when identifying the inertial parameters of a complex structure such as a humanoid robot, for which both structural \citep{ayusawa2013} and numerical \citep{pham1991essential} not identifiable parameters exists. 
In this later application, the enforcement of full physical consistency  will always be necessary to get meaningful results.


% \setlength{\arrayrulewidth}{1mm}
% \setlength{\tabcolsep}{18pt}
% \renewcommand{\arraystretch}{2.5}
 
% \newcolumntype{s}{>{\columncolor[HTML]{AAACED}} p{3cm}}

\definecolor{Gray}{gray}{0.85}
\newcolumntype{g}{>{\columncolor{Gray}}m}

\begin{table*}[ht]
\small
\caption{Inertial parameters identified with the different datasets and the different optimization problems.} 
\begin{center}
\begin{tabular}{ |g{0.6cm}|m{0.8cm} g{0.8cm}|m{0.8cm} g{0.8cm}|m{0.8cm} g{0.8cm}|m{0.8cm} g{0.8cm}|m{0.8cm} g{0.8cm}| }
\hhline{~----------}
\multicolumn{1}{c|}{}
 & \multicolumn{2}{|c|}{\cellcolor{white} \SI{10}{s}} & \multicolumn{2}{|c|}{\cellcolor{white} \SI{5}{s}} & \multicolumn{2}{|c|}{\cellcolor{white} \SI{2}{s}} & \multicolumn{2}{|c|}{\cellcolor{white} \SI{1}{s}} & \multicolumn{2}{|c|}{\cellcolor{white} \SI{0.5}{s}} \\
\cline{2-11}
\multicolumn{1}{c|}{}
 & $\mathbb{R}^{10}$ & $\ipspace$  & $\mathbb{R}^{10}$ & $\ipspace$   & $\mathbb{R}^{10}$ & $\ipspace$ & $\mathbb{R}^{10}$ & $\ipspace$   & $\mathbb{R}^{10}$ & $\ipspace$  \\
\hline 
$\mathbf{m}$ & 1.836 & 1.836 & 1.842 & 1.842 & 1.852 & 1.852 & 1.820 & 1.820 & 1.843 & 1.844 \\
\hline 
$\mathbf{mc_x}$ & 0.062 & 0.062 & 0.061 & 0.060 & 0.060 & 0.060 & 0.060 & 0.060 & 0.060 & 0.059 \\
\hline 
$\mathbf{mc_y}$ & 0.001 & 0.001 & 0.000 & 0.000 & 0.001 & 0.001 & 0.002 & 0.002 & 0.005 & 0.004 \\
\hline 
$\mathbf{mc_z}$ & 0.208 & 0.208 & 0.206 & 0.206 & 0.206 & 0.206 & 0.205 & 0.205 & 0.204 & 0.204 \\
\hline 
$\mathbf{I_{xx}}$ & \cellcolor[HTML]{FFCCCC} 0.580 & 0.215 & \cellcolor[HTML]{FFCCCC} 0.128 & 0.166 & 0.065 & 0.067 & 0.032 & 0.034 & 0.033 & 0.037 \\
\hline 
$\mathbf{I_{xy}}$ & \cellcolor[HTML]{FFCCCC} 0.593 & 0.012 & \cellcolor[HTML]{FFCCCC} {-}0.02 & 0.001 & 0.001 & 0.001 & 0.001 & 0.001 & 0.003 & 0.001 \\
\hline 
$\mathbf{I_{xz}}$ & \cellcolor[HTML]{FFCCCC} {-}0.54 & {-}0.06 & \cellcolor[HTML]{FFCCCC} {-}0.13 & {-}0.09 & {-}0.04 & {-}0.03 & {-}0.02 & {-}0.02 & {-}0.02 & {-}0.02 \\
\hline
$\mathbf{I_{yy}}$ & \cellcolor[HTML]{FFCCCC} 1.022 & 0.227 & \cellcolor[HTML]{FFCCCC} 0.125 & 0.216 & 0.066 & 0.086 & 0.036 & 0.042 & 0.035 & 0.039  \\
\hline 
$\mathbf{I_{yz}}$ & \cellcolor[HTML]{FFCCCC} 0.190 & 0.038 & \cellcolor[HTML]{FFCCCC} 0.026 & 0.001 & 0.006 & 0.003 & 0.002 & 0.001 & 0.000 & 0.000  \\
\hline 
$\mathbf{I_{zz}}$ & \cellcolor[HTML]{FFCCCC} -0.13 & 0.028 & \cellcolor[HTML]{FFCCCC} -0.00 & 0.050 & 0.007 & 0.014 & 0.008 & 0.009 & 0.008 & 0.008 \\
\hline 
\end{tabular}
\end{center}
\label{table:results}

Inertial parameters identified on $\mathbb{R}^{10}$ optimization manifold that are not fully physical consistent are highlighted.

Masses are expressed in \si{kg}, first moment of masses in \si{kg.m}, inertia matrix elements in \si{kg.m^2}.
\end{table*}





