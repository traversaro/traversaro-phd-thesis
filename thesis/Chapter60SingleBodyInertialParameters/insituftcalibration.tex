%%%%%%%%%%%%%%%%%%%%%%%%%%%%%%%%%%%%%%%%%%%%%%%%%%%%%%%%%%%%%%%%%%%%%%%%%%%%%%%%
%2345678901234567890123456789012345678901234567890123456789012345678901234567890
%        1         2         3         4         5         6         7         8

\documentclass[letterpaper, 10 pt, conference]{ieeeconf}  % Comment this line out if you need a4paper

%\documentclass[a4paper, 10pt, conference]{ieeeconf}      % Use this line for a4 paper

\IEEEoverridecommandlockouts                              % This command is only needed if 
                                                          % you want to use the \thanks command

\overrideIEEEmargins                                      % Needed to meet printer requirements.

% See the \addtolength command later in the file to balance the column lengths
% on the last page of the document

% The following packages can be found on http:\\www.ctan.org
\usepackage{graphicx} % for pdf, bitmapped graphics files
%\usepackage{epsfig} % for postscript graphics files
%\usepackage{mathptmx} % assumes new font selection scheme installed
%\usepackage{times} % assumes new font selection scheme installed
\usepackage{amsmath} % assumes amsmath package installed
\usepackage{amssymb}  % assumes amsmath package installed
% \usepackage{hyperref}
\usepackage{multirow}
\usepackage{colortbl}
\usepackage{subfig}

\newcommand{\x}{\ensuremath{\times}}

\title{\LARGE \bf
In Situ Calibration of Six-Axis Force-Torque Sensors \\ using Accelerometer Measurements*
}


\author{Silvio Traversaro and Daniele Pucci and Francesco Nori% <-this % stops a space
\thanks{*This paper was supported by the FP7 EU projects CoDyCo (No. 600716
ICT 2011.2.1 Cognitive Systems and Robotics) and Koroibot (No. 611909
ICT-2013.2.1 Cognitive Systems and Robotics).}% <-this % stops a space
\thanks{All authors belong to the Istituto Italiano di Tecnologia, RBCS department, Genova, Italy. Email: {\tt\small name.surname@iit.it}}%
}



\begin{document}

%\graphicspath{{./images/}}
\newcommand{\prop}[2]{\textbf{Proposition #1. }\textit{#2}}
\newcommand{\remark}[2]{\textbf{Remark #1. }\textit{#2}}
\newcommand{\lemma}[2]{\textbf{Lemma #1. }\textit{#2}}
\newcommand{\hp}[2]{\textbf{Assumption #1. }\textit{#2}}
\newcommand{\nrofsg}{6}
\newcommand{\calibmat}{\mathbf{C}}
\newcommand{\shapemat}{\mathbf{S}}
\newcommand{\rawval}{\mathbf{r}}
\newcommand{\senswrench}{{}^s\mathbf{w}}
\newcommand{\sensgrav}{{}^s\mathbf{g}}
\newcommand{\offsetwrench}{{}^\mathbf{o_w}}
\newcommand{\offsetraw}{\mathbf{o_r}}

\maketitle
\thispagestyle{empty}
\pagestyle{empty}

\newtheorem{hypothesis}{Assumption}

%%%%%%%%%%%%%%%%%%%%%%%%%%%%%%%%%%%%%%%%%%%%%%%%%%%%%%%%%%%%%%%%%%%%%%%%%%%%%%%%
\begin{abstract}
This paper proposes techniques to calibrate six-axis force-torque sensors that can be performed \emph{in situ}, i.e., without removing the sensor from
the hosting 
system. 
We assume that the force-torque sensor is attached to a rigid body equipped with an accelerometer. Then, the proposed calibration 
technique uses the measurements of the accelerometer, but
% do not 
requires neither the knowledge of the inertial parameters 
% mass and the center of mass 
% of the rigid body 
nor the orientation
% , with respect to gravity, 
of the rigid body. 
The proposed method exploits the geometry induced by the model between the raw measurements of the 
sensor and the corresponding force-torque. 
% Furthermore, it does not require the knowledge of the mass and center of mass of the body to which the force-torque sensor is attached.
The validation of the approach is performed by calibrating two six-axis force-torque sensors of the iCub humanoid 
robot.
\end{abstract}

%%%%%%%%%%%%%%%%%%%%%%%%%%%%%%%%%%%%%%%%%%%%%%%%%%%%%%%%%%%%%%%%%%%%%%%%%%%%%%%%
\section{INTRODUCTION}
The importance of sensors in a control loop goes without saying.
% ~\cite{SomeoneIfoundAndThatInoLongerFind}.
Measurement devices, however, can seldom be used \emph{sine die} without being subject to periodic calibration procedures.
Pressure transducers, gas sensors, and thermistors are only a few examples of measuring devices that need to be calibrated periodically for providing 
the user with precise and robust measurements.
Of most importance, calibration procedures may require to move the sensor from the hosting system to specialized laboratories, which are equipped with
the tools for performing the calibration of the measuring device. This paper presents techniques to calibrate 
strain gauges six-axis force-torque sensors \emph{in situ}, i.e. without the need of removing the sensor from the hosting system.
The proposed calibration method is particularly suited for robots with embedded six-axis force-torque sensors installed within limbs~\cite{Fumagalli2012}. 

Calibration of six-axis force-torque sensors  has long attracted the attention of the robotic community~\cite{braun2011}. 
The commonly used model for predicting the force-torque 
from the raw measurements of the sensor is an affine model.
This model is sufficiently accurate 
since these sensors are mechanically designed and mounted so that the strain deformation is (locally) linear with respect to the applied forces and torques.
Then, the calibration of the sensor aims at determining the two components of this model, i.e. a six-by-six matrix and a six element vector.
These two components are usually referred to as the sensor's \emph{calibration matrix} and \emph{offset}, respectively.
In standard operating conditions, relevant changes in the calibration matrix may occur in months.
As a matter of fact, manufacturers such as ATI~\cite{atimanual} and Weiss Robotics~\cite{kms40manual} 
recommend to calibrate force-torque sensors at least once a year.
Preponderant changes in the sensor's offset can occur in hours, however, and this in general requires to estimate the 
offset before using the  sensor.
The offset models the sensitivity of the semiconductor strain gauges with respect to temperature.

\begin{figure}
\vspace{0em}
\centering
\includegraphics[width=0.225\textwidth]{images/leg.pdf};
\caption{iCub's leg with the two force/torque sensors and an additional accelerometer.}
\label{fig:iCubLeg}
\end{figure} 
Classical techniques for determining the offset of a force-torque sensor exploit the aforementioned affine model between 
the raw measurements and the load attached to the sensor. In particular, if no load is applied to the measuring device, the output of the sensor corresponds to 
the sensor's offset. This offset identification procedure, however, cannot be always performed since it may require to take the hosting system apart
in order to unload the force-torque sensor. Another widely used technique for offset identification is to find two sensor's orientations that induce
equal and opposite loads with respect to the sensor. Then, by summing up the raw measurements associated with these two orientations, one can estimate
the sensor's offset. The main drawback of this technique is that the positioning of the sensor at these opposite configurations may require to move
the hosting system beyond its operating domain.

Assuming a preidentified offset, non-in situ identification of the calibration matrix is classically performed by exerting on  
the sensor a set of force-torques known \emph{a priori}. This usually requires to place 
sample masses at precise relative positions with respect to the sensor. Then, by comparing
the known gravitational force-torque with that measured by the sensor, 
one can apply linear least square techniques to identify the sensor's calibration matrix.
For accurate positioning of the sample masses, the use of robotic positioning devices 
has also been proposed in the specialized 
literature~\cite{uchiyama1991systematic}~\cite{watson1975pedestal}.
%Clearly, the more numerous the sample masses, the smaller the identification error. 

To reduce the number of sample masses,
one can apply constrained forces, e.g. constant norm forces, to the measuring device.
Then these constrains can be exploited during the computations for identifying the calibration matrix~\cite{voyles1997shape}.
To avoid the use of added masses, one can use a supplementary already-calibrated measuring device that measures 
the force-torque exerted on the sensors~\cite{faber2012force}~\cite{oddo2007}.
On one hand, this calibration technique avoids the problem of precise positioning of the added sample masses.
On the other hand, 
the supplementary sensor may not always be available.
All above techniques, however, cannot be performed in situ, thus  being usually time consuming and expensive.


To the best of our knowledge, the first \emph{in situ} calibration method for force-torque sensors was proposed 
in \cite{shimanoroth}. But this method  exploits the topology of a specific kind of manipulators, which are equipped with
joint torque sensors then leveraged during the estimation. 
A recent result~\cite{Gong2013} proposes another \emph{in situ} calibration technique for six-axis force-torque sensors. 
The technical soundness of this work, however, is not clear to us. In fact, we show that a necessary condition for identifying the calibration matrix
is to take measurements for at least three different added masses, and this requirement was not met by the algorithm~\cite{Gong2013}.
Another in situ calibration technique for force-torque sensors can be found in \cite{roozbahani2013novel}.
But the use of supplementary already-calibrated force-torque/pressure sensors impairs this technique for the reasons we have discussed before. 




This paper presents in situ calibration techniques for six-axis force-torque sensors using accelerometer measurements.
The proposed method exploits the geometry induced by the affine model between the raw measurements and the gravitational force-torque applied to the sensor. 
In particular, it relies upon the properties that all gravitational raw measurements belong to a three-dimensional space, and that in this space they form an ellipsoid.
Then, the contribution of this paper is twofold. We first propose a method for estimating the sensor's offset, and then a procedure for identifying
the calibration matrix. The latter is independent from the former, but requires to add sample masses to the rigid body attached to the sensor. Both methods are independent from the inertial characteristics 
of the rigid body attached to the sensor. 
The proposed algorithms are validated on the iCub platform by calibrating
two force-torque sensors embedded in the robot~leg.

The paper is organized as follows. Section~\ref{sec:background} presents the notation used in the paper and the problem statement with the assumptions. 
Section~\ref{method} details the proposed method for the calibration of six-axis force-torque sensors. 
Validations of the approach are presented in Section~\ref{experiments}.
Remarks and perspectives conclude the paper.




\section{BACKGROUND}
\label{sec:background}

\subsection{Notation}
The following notation is used throughout the paper.
\begin{itemize}
 \item The set of real numbers is denoted by $\mathbb{R}$. Let $u$ and $v$ be two $n$-dimensional column vectors of real numbers, i.e. $u,v \in \mathbb{R}^n$, 
 their inner product is denoted as $u^\top v$, with ``$\top$'' the transpose operator.
\item Given $u \in \mathbb{R}^3$, $u \times$ denotes the skew-symmetric matrix-valued operator associated with the cross product in 
  $\mathbb{R}^3$.
%  \item Given a time function $f(t) \in \mathbb{R}^n$, its first and second order time derivative are denoted by $\dot{f}(t)$ and $\ddot{f}(t)$,
%  respectively. Given a function $f(\cdot)$ of several variables, its gradient
% w.r.t. some of them, say $x$, is the row vector denoted as $ \partial_x f$.
 \item The Euclidean norm of either a vector or a matrix of real numbers is denoted by $|\cdot |$.
%  \item $b_i \in \mathbb{R}^n$ denotes the column vector of $n$ zeros but the $\imath$th coordinate, which is equal to one. 
\item $I_n \in \mathbb{R}^{n \times n}$ denotes the identity matrix of dimension~$n$; 
$0_n \in \mathbb{R}^n$ denotes the zero column vector of dimension~$n$; $0_{n \times m} \in \mathbb{R}^{n \times m}$ denotes the zero matrix of dimension~$n \times m$.
\item The vectors $e_1,e_2,e_3$ denote the canonical basis of $\mathbb{R}^3$.
\item Let $\mathcal{I} = \{O;e_1,e_2,e_3\}$ denote a 
fixed inertial frame with respect to (w.r.t.) which the sensor's absolute orientation is measured. 
Let $\mathcal{S} = \{O';i,j,k\}$ denote a frame attached to the sensor, where the matrix $T := (i,j,k)$ is a rotation matrix
whose column vectors are the vectors
of coordinates of $i,j,k$ expressed in the basis of  $\mathcal{I}$. 
\item The sensor's orientation w.r.t. $\mathcal{I}$ is characterized by the rotation matrix $T$. Given a vector of coordinates $\bar{x} \in \mathbb{R}^3$
expressed w.r.t. $\mathcal{I}$, we denote with $x$ the same vector expressed w.r.t. $\mathcal{S}$, i.e. $\bar{x} = Tx$.
\item Given $A \in \mathbb{R}^{n \times m}$ and $B \in \mathbb{R}^{p \times q}$, we denote with $\otimes$ the Kronecker product $A \otimes B \in \mathbb{R}^{np \times mq}$.
\item Given $X \in \mathbb{R}^{m \times p}$, $\text{vec}(X) \in \mathbb{R}^{nm}$ denotes the column vector obtained by stacking the columns of the matrix~$X$. 
In view of the definition of $\text{vec}(\cdot)$, it follows that \begin{equation}\label{eq:kroneckerVec} \text{vec}(AXB) = \left( B^{\top} \otimes A \right) \text{vec}(X).\end{equation}

% \item For the sake of conciseness, a vector 
% \[{\bf{x}} = \bar{x}_1 {\bf{i_0}} + \bar{x}_2 {\bf{j_0}} + \bar{x}_3 {\bf{k_0}} = x_1 {\bf{i}} + x_2 {\bf{j}} +x_3 {\bf{k}}\] 
% is written as 
% \[{\bf{x}} = ({\bf{i_0}},{\bf{j_0}},{\bf{k_0}})\bar{x} = ({\bf{i}},{\bf{j}},{\bf{k}})x\] 
% where the two vectors of coordinates $\bar{x}, x \in \mathbb{R}^3$ represent the vector $\bf{x}$ when expressed either w.r.t. the inertial
% frame $\mathcal{I}$ or w.r.t. the sensor frame $\mathcal{S}$, respectively. As a consequence, 
% \[ \bar{x} = R x. \]
\end{itemize}

\subsection{Problem statement and assumptions}
We assume that the model for predicting the force-torque (also called wrench)  
% applied to the sensor 
from the raw measurements is an affine model, i.e. 
% \begin{equation}
% \rawval = \shapemat \hspace{0.2em} \senswrench -  {o_r}
% \end{equation} 
% Consequently, the output of the sensor can be computed as $\nrofsg$ 
%(with $m \geq 6$) 
% raw strain gauges values is :
\begin{equation}
\label{wrenchInSensorCoordinates}
w =  C ( r - o),
% =  {C} \left( \rawval -  {o_r} \right)
\end{equation} 
where
% \begin{itemize}
% \item 
$ {w} \in \mathbb{R}^{6}$ is the wrench exerted on the sensor expressed in the sensor's frame,
% \item 
$r \in \mathbb{R}^{\nrofsg}$ is the raw output of the sensor,
% \item 
$ {C} \in \mathbb{R}^{6 \times \nrofsg}$ is the
%full row rank 
invertible
calibration matrix, and
% \item 
${o} \in \mathbb{R}^6$ is the sensor's offset.
The calibration matrix and the offset are assumed to be constant.
% \end{itemize}

We assume that the sensor is attached to a rigid body
% mechanical structure --~such as a robot manipulator, 
% see Figure~\ref{photoOfTheIcubLegOrArmWhereYouSeeTheMachanincalChainAttachedToIt}~-- and that
% this structure posses $n$ degrees of freedoms. The associated configuration of this structure can be
% then characterized by an $n$-dimensional column vector $q \in \mathbb{R}^n$. 
% The mechanical structure has a total 
of 
(constant) mass $m~\in~\mathbb{R}^+$ and with a center of mass whose position
w.r.t. the sensor frame $\mathcal{S}$ is characterized the vector $c \in \mathbb{R}^3$.

% \[{\bf{OC}}(q) = ({\bf{i_0}},{\bf{j_0}},{\bf{k_0}})\bar{c}(q) = ({\bf{i}},{\bf{j}},{\bf{k}})c(q).\]
% \[\bar{c} = Tc,\]
% with $\bar{c},c \in \mathbb{R}^3$ the vector of the center of mass expressed w.r.t. the inertial and sensor frame, respectively.
The gravity force applied to the 
% structure 
body
is  given by 
\begin{IEEEeqnarray}{RCL}
 \label{eq:g}
 m\bar{g} = mTg ,
\end{IEEEeqnarray}
with $\bar{g},g \in \mathbb{R}^3$ the gravity acceleration expressed w.r.t. the inertial and sensor frame, respectively. The gravity acceleration $\bar{g}$ is constant, so the vector $g$ does not have a constant direction, 
but has a constant norm.

Finally, we make the following main assumption.

\hp{}{
 The raw measurements $r$ are taken for static configurations of the 
 rigid body
%  mechanical structure 
 attached to the sensor, i.e. the angular velocity of the frame $\mathcal{S}$ is always zero.
 Also, the gravity acceleration $g$ is measured by an accelerometer installed on the rigid body.
%  $\tfrac{d q}{dt} \equiv 0$. 
  Furthermore, no external force-torque, but the gravity force, acts on the rigid body. Hence
}
\begin{IEEEeqnarray}{RCL}
\label{eq:staticWrench}
w &=& 
M(m,c) g, \IEEEyessubnumber  \\
M(m,c) &:=& m
\begin{pmatrix}
I_3 \\
c \times 
\end{pmatrix}. \label{matrixM}  \IEEEyessubnumber 
% = 
% \mathbf{M} \hspace{0.2em}{}^s\mathbf{g}
% =
% \mathbf{M} \hspace{0.2em}
% {}^s\rawval_w {}^w \mathbf{g}
\end{IEEEeqnarray} 

\remark{}{ We implicitly assume that the accelerometer frame is aligned with the force-torque sensor frame. 
This is a convenient, but non necessary, assumption. 
In fact, if the relative rotation between the sensor frame $\mathcal{S}$ and the accelerometer frame is unknown, 
it suffices to consider the accelerometer frame as the sensor frame $\mathcal{S}$.
}

Under the above assumptions, what follows proposes a new method for estimating the sensor's offset $o$ 
and for identifying the sensor's calibration matrix $C$ without the need of removing the sensor from the 
hosting system.


\input{sections/method.tex}

\input{sections/experimental.tex}

\section{Conclusions}
\label{conclusions}
In this paper, we addressed the problem of calibrating six-axis force-torque sensors in situ by using the accelerometer measurements. 
The main point was to highlight the geometry behind the gravitational
raw measurements of the sensor, which can be shown to belong to a three-dimensional affine space, and more precisely to a three-dimensional ellipsoid.
Then, we propose a method to identify first the sensor's offset, and then the sensor's calibration matrix. The latter method requires to add sample masses
to the body attached to the sensor, but is independent from the mass and the center of mass of this body. We show that a necessary condition to identify the sensor's calibration matrix is to collect data for more than two 
sample masses. The validation of the method was performed by calibrating the two force-torque sensors embedded in the iCub leg.

The main assumption under the proposed algorithm is that the measurements were taken for static configurations of the rigid body attached to the sensor.
Then, future work consists in weakening this assumption, and developing calibration procedures that hold even for dynamic motions of the rigid body. This
extension requires to use the gyros measurements. In addition, comparisons of the proposed method versus existing calibration 
techniques is currently being investigated, and will be presented in a forthcoming publication.

%\addtolength{\textheight}{-12cm}   % This command serves to balance the column lengths
                                  % on the last page of the document manually. It shortens
                                  % the textheight of the last page by a suitable amount.
                                  % This command does not take effect until the next page
                                  % so it should come on the page before the last. Make
                                  % sure that you do not shorten the textheight too much.




%%%%%%%%%%%%%%%%%%%%%%%%%%%%%%%%%%%%%%%%%%%%%%%%%%%%%%%%%%%%%%%%%%%%%%%%%%%%%%%%

\bibliographystyle{IEEEtran}
\bibliography{bibl}


\end{document}
