\section{Introduction}
% KNOWN
A fundamental problem in controlling torque-actuated robots is the accurate modeling of their dynamics. Depending on the performed task (e.g. control, simulation, contact detection) we can distinguish two possible approaches \citep{handbookident}: in \emph{structural modeling} the interest is on identifying the real inertial parameters of the robot, while in \emph{predictive modeling} the interest is only on replicating the input-output behavior of the system, the input and output being some measured quantities. 
In structural modeling the usual approach is to excite the robot with trajectories chosen so as to be the optimal for identifying the identifiable (i.e. base) parameters \citep{armstrong}. 
In predictive modeling, the only concern is to accurately model the input-output response of the dynamical system. This has a significant implication: the interest is not in estimating the ``real'' parameters, but in getting parameters capable of generalizing predictions across the whole work space. 
Interestingly, within this context different regression techniques can be adopted, ranging from parametric \citep{handbookident}, semi-parametric \citep{Nguyen-Tuong2010} and machine learning approaches \citep{Fumagalli_FMLILR_2010}. A common task that falls within the predictive modeling category is learning inverse dynamics: inputs are positions, velocities and accelerations while outputs are joint torques. 

In this chapter we tackle a problem that lies in between \emph{structural} and \emph{predictive modeling}. We aim to relax the modeling assumptions in the procedure illustrated in Chapter~\ref{chap:extForceAndJntTorqueEstimation} to estimate joint torques from embedded 6-axis force/torque sensors. This estimation procedure allows us to implement torque control on robots without joint torque sensing. Since most (humanoid) robots are not equipped with joint torque sensors, but have 6-axis F/T sensors, this approach opens the possibility to implement inverse-dynamics control on these ``old-generation'' robots. Moreover, this is  interesting also for new-generation robots, which could be easily equipped with 6-axis F/T sensors, without going through the hassle of redesigning the joints to include torque sensing. 

The main drawback of this method is that it relies on the inertial parameters to estimate the joint torques. The goal of this paper is to understand if and to what extent this knowledge is necessary and if we can partially retrieve it through identification procedures similar to the one proposed in \citep{handbookident}. The major technical obstacle lies in the following consideration. We can use F/T measurements to estimate certain inertial parameters (known in literature as \emph{base parameters}): what is the relationship between these parameters and the ones used in \citep{Fumagalli2012} to estimate (internal) joint torques and (external) contact forces? In this framework non-parametric techniques have limited appeal and therefore we pursue a parametric approach. 

Consistently with the rest of the thesis, in this chapter we will discuss the problem of estimating inertial parameters in the context of free floating robots. In particular we will discuss the case of iCub and its specific set of sensors, as introduced in Section~\ref{sec:icub}.



